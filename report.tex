\documentclass{article}
\usepackage{graphicx}

\begin{document}

\title{CM30225 Parallel Computing \\ Assessed Courseork Assignment 1}
\author{}

\maketitle

\section{Parralisation Technique}
In order to parallelise the problem I decided to spilt the matrix up into rows
then give each thread a number of these rows. More specifically each thread is
given a starting row, which is applies relaxation too, then adds the number of
threads to the starting row to get the next row to compute on. So if 4 threads
are used on a 14, the two end rows are fixed and don't require relaxation,
row matrix the rows are split up like so:

\begin{center}
\begin{tabular}{ |c|c|c|c|c|c|c|c|c|c|c|c|c|c|c| }
 \hline
 rowNumber & 1 & 2 & 3 & 4 & 5 & 6 & 7 & 8 & 9 & 10 & 11 & 12 & 13 & 14 \\
 thread &  & 1 & 2 & 3 & 4 & 1 & 2 & 3 & 4 & 1 & 2 & 3 & 4 &  \\
 \hline
\end{tabular}
\end{center}

The threads are then all syncronised using a barrier so after all rows have been
computed the read and write matrixs are swapped and the computation continues.\\~\\
A quick note is that the program will actual use the number of thread plus the
main thread, it was decided not to change this as the main thread just will be
desheduled while it waits for all the other threads to return so won't effect
the investigation significantly.

\section{Avoiding Race Conditions}
There are a few places where race conditions are possible:

\begin{enumerate}
\item Firstly two matrixs are used, one to read from and then one to write to.
This avoids threads trying to write and read to the same cell simultaniously,
similtanious reads are possible but not a problem. Also if one matrix was used without
a lock the program would yeild diffrent results each time depending on if a cell
was computed before of after its neighbours are.
\item Once a thread has finished it sets a global variable ``cont" to 1 if the
precision was not met. this variable does not need a lock as if one thread resets
it after it doesn't matter as its still 1.
\end{enumerate}

\end{document}
